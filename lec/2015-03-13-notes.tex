\documentclass[12pt, leqno]{article}
\usepackage{amsfonts}
\usepackage{amsmath}
\usepackage{fancyhdr}
\usepackage{hyperref}
\usepackage{tikz}
\usepackage{pgfplots}
\usepackage{listings}

\newcommand{\bbR}{\mathbb{R}}
\newcommand{\bbC}{\mathbb{C}}

\newcommand{\hdr}[2]{
  \pagestyle{fancy}
  \lhead{Bindel, Spring 2015}
  \rhead{Numerical Analysis (CS 4220)}
  \fancyfoot{}
  \begin{center}
    {\large{\bf #1}} \\
    Due: #2
  \end{center}
  \lstset{language=matlab,columns=flexible}
}

\newcommand{\phdr}[1]{
  \pagestyle{fancy}
  \lhead{Bindel, Spring 2015}
  \rhead{Numerical Analysis (CS 4220)}
  \fancyfoot{}
  \begin{center}
    {\large{\bf #1}}
  \end{center}
  \lstset{language=matlab,columns=flexible}
}


\newcommand{\calK}{\mathcal{K}}
\newcommand{\calP}{\mathcal{P}}
\newcommand{\calR}{\mathcal{R}}

\begin{document}
\hdr{2015-03-13}

\section*{From Lanczos to CG}

In the last lecture, we developed the Lanczos iteration, which
for a symmetric matrix $A$ implicitly generates the decomposition
\[
  A Q^{(k)} = Q^{(k)} T^{(k)} + \beta_k q_{k+1}
\]
where $T^{(k)}$ is a tridiagonal matrix with $\alpha_1, \ldots,
\alpha_k$ on the diagonal and $\beta_1, \ldots, \beta_{k-1}$ on
the subdiagonal and superdiagonal.  The columns of $Q^{(k)}$ form
an orthonormal basis for the Krylov subspace $\calK_{k}(A,b)$,
and are a numerically superior alternative to the power basis.
We now turn to using this decomposition to solve linear systems.

The {\em conjugate gradient} algorithm can be characterized as
a method that chooses an approximation
$\tilde{x}^{(k)} \in \calK_k(A,b)$ by minimizing the energy
function
\[
  \phi(z) = \frac{1}{2} z^T A z - z^T b
\]
over the subspace.  Writing $\tilde{x}^{(k)} = Q^{(k)} u$,
and using the fact that
\begin{align*}
  (Q^{(k)})^T A Q^{(k)} & = T^{(k)} \\
  (Q^{(k)})^T b &= \|b\| e_1
\end{align*}
we have
\[
  \phi(Q^{(k)} u) = \frac{1}{2} u^T T^{(k)} u - \|b\| u^T e_1.
\]
The stationary equations in terms of $u$ are then
\[
  T^{(k)} u = \|b\| e_1.
\]

In principle, we could solve find the CG solution by forming and
solving this tridiagonal system at each step, then taking an
appropriate linear combination of the Lanczos basis vectors.
Unfortunately, this would require that we keep around the Lanczos
vectors, which eventually may take quite a bit of storage.
This is essentially what happens in methods like GMRES, but
for the method of conjugate gradients, we have not yet exhausted
our supply of cleverness.  It turns out that we can derive a short
recurrence relating the solutions at consecutive steps and their
residuals.  There are several different ways to this recurrence:
one can work from a factorization of the nested tridiagonal matrices
$T^{(k)}$, or work out the recurrence based on the optimization
interpretation of the problem (this leads to the name ``conjugate
gradients'').

\section*{Practical Matters}

CG is popular for several reasons:
\begin{itemize}
\item
  The only thing we need $A$ for is to form matrix-vector products.
  An implicit representation of $A$ may be sufficient for this
  (and is in some cases more convenient than an explicit version).
\item
  Because it involves a short recurrence, CG can be run for many steps
  without an excessive amount of storage or other overheads involving
  looking over many past vectors.  This is not true of all other CG
  methods.
\item
  Convergence is faster than with stationary methods.
\end{itemize}
This last point requires some clarification, and will take up most of
the rest of our discussion.

When we started talking about Krylov subspaces, we described searching
over the spaces $\calK_m(M^{-1} A, M^{-1} b)$, where $M$ is the matrix
appearing in the splitting for some stationary iteration.  When we
derived CG, though, the matrix $M$ disappeared.  In fact, it
disappeared only to keep the presentation as uncluttered as I could
manage.  In practice, CG and other Krylov subspace methods are
typically used with a {\em preconditioner} $M$, and one looks over
$\calK_m(M^{-1} A, M^{-1} b)$ for solutions.  The preconditioner
should have the following properties
\begin{itemize}
\item
  As in a splitting for a stationary method, it should be easy to
  apply $M^{-1}$.  There is a tradeoff here: the quality of the
  subspace and the efficiency with which $M^{-1}$ can be applied may
  be in conflict.  Note that solving systems involving $M$ is the only
  way in which $M$ is used (just as applying $A$ is the only way in
  which $A$ is used).
\item
  The preconditioner does not need to correspond to a convergent
  stationary iteration, but $M^{-1}$ should ``look like'' $A^{-1}$
  in that $M^{-1} A$ should have eigenvalues in clusters.  If all the
  eigenvalues are the same, preconditioned CG (or other preconditioned
  Krylov subspace methods) will converge in one step.
\item
  For CG, the preconditioner must be symmetric and positive definite.
\end{itemize}

Are preconditioners really necessary?  For problems coming from PDE
discretizations or computations on networks, there is sometimes an
intuitive way to see why the answer is ``yes'' if one wants fast
convergence.  Consider, for example, the model tridiagonal matrix $T$
in $\bbR^{N \times N}$, and imagine we want to solve $Tx = e_1$.  The
last component of $x$ is not that tiny, but notice that all the
vectors in a Krylov subspace $\calK_m(T, e_1)$ are zero for every
index after $m$.  Therefore, there is no way a method that explores
these Krylov subspaces will be close to converged for $m < N$
iterations.  Given that we know how to solve this tridiagonal system
in $O(N)$ time with Gaussian elimination, this is a bit disheartening!
The issue is that method simply doesn't move information through the
mesh fast enough to achieve rapid convergence.

One way of deriving preconditioners is to look at the splittings used
in classical stationary iterations.  Another approach is incomplete
factorization -- carry out Gaussian elimination or Cholesky, but
throw away nonzeros that are small or appear in inconvenient places.
However, the best preconditioners are often specific to particular
applications.  Some examples are:
\begin{itemize}
\item
  For discretizations of PDEs with variable coefficients, we might use
  a preconditioner based on a much more regular PDE (e.g. with
  constant coefficients and a regular geometry).  The preconditioner
  solve could potentially be applied by fast transform methods.
\item
  Multigrid preconditioners take advantage of the fact that one can
  often discretize a continuous problem using coarser or finer meshes,
  and the coarse meshes can be used to suppress parts of the error
  that are hard to reach by applying standard stationary methods to
  fine grids.
\item
  Domain decomposition preconditioners split the problem into
  subproblems.  If the subproblems are treated completely
  independently, domain decomposition looks like a block version of
  Jacobi iteration; but if the subproblems overlap, or if they are
  coupled together in some other way, one gets any of a variety of
  interesting methods (additive and multiplicative Schwarz, FETI,
  BDDC, etc).
\end{itemize}
Frameworks like PETSc and Trilinos provide both standard iterative
solvers like CG and a menu of different preconditioners.  Choosing the
right preconditioner is in general hard, and choosing the parameters
that determine the detailed behavior is hard.


\end{document}
