\documentclass[12pt, leqno]{article}
\usepackage{amsfonts}
\usepackage{amsmath}
\usepackage{fancyhdr}
\usepackage{hyperref}
\usepackage{tikz}
\usepackage{pgfplots}
\usepackage{listings}

\newcommand{\bbR}{\mathbb{R}}
\newcommand{\bbC}{\mathbb{C}}

\newcommand{\hdr}[2]{
  \pagestyle{fancy}
  \lhead{Bindel, Spring 2015}
  \rhead{Numerical Analysis (CS 4220)}
  \fancyfoot{}
  \begin{center}
    {\large{\bf #1}} \\
    Due: #2
  \end{center}
  \lstset{language=matlab,columns=flexible}
}

\newcommand{\phdr}[1]{
  \pagestyle{fancy}
  \lhead{Bindel, Spring 2015}
  \rhead{Numerical Analysis (CS 4220)}
  \fancyfoot{}
  \begin{center}
    {\large{\bf #1}}
  \end{center}
  \lstset{language=matlab,columns=flexible}
}


\begin{document}
\hdr{2015-03-06}

\section*{Direct to Iterative}

For the past few weeks, we have discussed {\em direct} methods for
solving linear systems and least squares problems.  These methods
typically involve a factorization, such as LU or QR, that reduces the
problem to a triangular solve using forward or backward substitution.
These methods run to completion in a fixed amount of time, and are
backed by reliable software in packages like LAPACK or UMFPACK.

There are a few things you need to know to be an informed {\em user}
(not developer) of direct methods:
\begin{itemize}
\item
  You need some facility with matrix algebra, so that you know how to
  manipulate matrix factorizations and ``push parens'' in order to
  compute efficiently.
\item
  You need to understand the complexity of different factorizations,
  and a little about how to take advantage of common matrix structures
  (e.g. low-rank structure, symmetry, orthogonality, or sparsity) in
  order to effectively choose between factorizations and algorithms.
\item
  You need to understand a little about conditioning and the
  relationship between forward and backward error.  This is important
  not only for understanding rounding errors, but also for
  understanding how other errors (such as measurement errors) can
  affect a result.
\end{itemize}
It's also immensely helpful to understand a bit about how the methods
work in practice.  On the other hand, you are unlikely to have to
build your own dense Gaussian elimination code with blocking for
efficiency; you'll probably use a library routine instead.  It's more
important that you understand the ideas behind the factorizations, and
how to apply those ideas to use the factorizations effectively in
applications.

Iterative methods are different, in that they provide more room for
clever, application-specific twists and turns.  An iterative method
for solving the linear system $Ax = b$ produces a series of guesses
\[
  \hat{x}^1, \hat{x}^2, \ldots \rightarrow x.
\]
The goal of the iteration is not always to get the exact answer as
fast as possible; it is to get a good enough answer, fast enough to be
useful.  The rate at which the iteration converges to the solution
depends not only on the nature of the iterative method, but also on
the structure in the problem.  The picture is complicated by the fact
that different iterations cost different amounts per step, so a
``slowly convergent'' iteration may in practice get an adequate
solution more quickly than a ``rapidly convergent'' iteration, just
because each step in the slowly convergent iteration is so cheap.

As with direct methods, though, sophisticated iterative methods are
constructed from simpler building blocks.  In this lecture, we set up
one such building block: stationary iterations.

\section*{Stationary Iterations}

A stationary iteration for the equation $Ax = b$ is typically
associated with a {\em splitting} $A = M-N$, where $M$ is a matrix
that is easy to solve (i.e. a triangular or diagonal matrix) and $N$
is everything else.  In terms of the splitting, we can rewrite
$Ax = b$ as
\[
  Mx = Nx + b,
\]
which is the fixed point equation for the iteration
\[
  Mx^{k+1} = Nx^{k} + b.
\]
If we subtract the fixed point equation from the iteration equation,
we have the error iteration
\[
  M e^{k+1} = N e^k
\]
or
\[
  e^{k+1} = R e^k, \quad R = M^{-1} N.
\]
We've already seen one example of such an iteration (iterative
refinement with an approximate factorization); in other cases,
we might choose $M$ to be the diagonal part of $A$ (Jacobi iteration)
or the upper or lower triangle of $A$ (Gauss-Seidel iteration).
We will see in the next lecture that there is an alternate
``matrix-free'' picture of these iterations that makes sense in the
context of some specific examples, but for analysis it is often best
to think about the splitting picture.

\section*{Convergence: Norms and Eigenvalues}

We consider two standard approaches to analyzing the convergence of a
stationary iteration, both of which revolve around the error iteration
matrix $R = M^{-1} N$.  These approaches involve taking a norm
inequality or using an eigenvalue decomposition.  The first approach
is often easier to reason about in practice, but the second is
arguably more informative.

For the norm inequality, note that if $\|R\| < 1$ for some operator
norm, then the error satisfies
\[
  \|e^{k+1}\| \leq \|R\| \|e^k\| \leq \|R\|^k \|e^0\|.
\]
Because $\|R\|^k$ converges to zero, the iteration eventually
converges.  As an example, consider the case where $A$ is strictly row
diagonally dominant and $M$ is the diagonal.  In that case,
$\|R\|_\infty = \|M^{-1} N\|_\infty < 1$.  Therefore, the infinity
norm of the error is monontonically decreasing\footnote{%
  In finite-dimensional spaces, there is a property of ``equivalence
  of norms'' that says that convergence in one norm implies
  convergence in any other norm; however, this does {\em not} mean
  that monotone convergence in one norm implies monotone convergence
  in any other norm.}

Bounding by one the infinity norm (or two norm, or one norm) of the
iteration matrix $R$ is {\em sufficient} to guarantee convergence,
but not {\em necessary}.  In order to completely characterize when
stationary iterations converge, we need to turn to an eigenvalue
decomposition.  Suppose $R$ is diagonalizable, and write the
eigendecomposition as
\[
  R = V \Lambda V^{-1}.
\]
Now, note that $R^k = V \Lambda^k V^{-1}$, and therefore
\[
  \|e^k\| = \|R^k e^0\| = \|V \Lambda^k V^{-1} e^0\| \leq \kappa(V)
  \rho(R)^k \|e^0\|,
\]
where $\rho(R)$ is the {\em spectral radius} of $R$, i.e.
\[
  \rho(R) = \max_{\lambda \mbox{ an eig}} |\lambda|,
\]
and $\kappa(V) = \|V\| \|V^{-1}\|$.  For a diagonalizable matrix,
convergence of the iteration happens if and only if the spectral
radius of $R$ is less than one.  {\em But} that statement ignores
the condition number of the eigenvector matrix!  For highly
``non-normal'' matrices in which the condition number is large,
the iteration may appear to make virtually no progress for many steps
before eventually it begins to converge at the rate predicted by
the spectral radius.  This is consistent with the bounds that we
can prove, but often surprises people who have not seen it before.

\end{document}
