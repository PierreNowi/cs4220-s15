\documentclass[12pt, leqno]{article}
\usepackage{amsfonts}
\usepackage{amsmath}
\usepackage{fancyhdr}
\usepackage{hyperref}
\usepackage{tikz}
\usepackage{pgfplots}
\usepackage{listings}

\newcommand{\bbR}{\mathbb{R}}
\newcommand{\bbC}{\mathbb{C}}

\newcommand{\hdr}[2]{
  \pagestyle{fancy}
  \lhead{Bindel, Spring 2015}
  \rhead{Numerical Analysis (CS 4220)}
  \fancyfoot{}
  \begin{center}
    {\large{\bf #1}} \\
    Due: #2
  \end{center}
  \lstset{language=matlab,columns=flexible}
}

\newcommand{\phdr}[1]{
  \pagestyle{fancy}
  \lhead{Bindel, Spring 2015}
  \rhead{Numerical Analysis (CS 4220)}
  \fancyfoot{}
  \begin{center}
    {\large{\bf #1}}
  \end{center}
  \lstset{language=matlab,columns=flexible}
}


\begin{document} \hdr{PS 7}{Mon, Apr 13}

\paragraph*{1: By the Book}
Section 9.4, Problem 6.  For part (b), it may help to refer to
a homework problem earlier in the semester!

\paragraph*{2: Naive Newton}
Consider the nonlinear system of equations
\begin{align*}
  x^2 + xy^2 &= 9 \\
  3x^2y - y^3 &= 4
\end{align*}
Fill in the following Newton iteration code:
\begin{lstlisting}
  for k = 1:20
    F = % Your code here
    if norm(F) < rtol
      break;
    end
    J = % Your code here
    dx = J\F;
    x = x-dx;
  end
\end{lstlisting}
Run your code with an initial guess of $(1,1)$ and a residual norm
tolerance of $10^{-12}$.  Do you see quadratic convergence?

\paragraph*{3: Continue with Care}
Consider the boundary value problem
\begin{align*}
  v''(x) + \gamma \exp(v(x)) &= 0, \quad 0 < x < 1 \\
  v(0) = v(1) &= 0
\end{align*}
discretized via finite differences on a mesh with 100 equally spaced
points; see example 9.3 in the book.
\begin{itemize}
\item
  Write a code to find $v$ for
  a range of $\gamma$ values from 1 to 3.5 (use
  {\tt gammas = linspace(1,3.5)} to generate the mesh).  For the first
  value of $\gamma$, you should use an initial guess of $v = 0$;
  for subsequent values, use the value of $v$ at the previous
  $\gamma$ step.  Plot all your solutions together on a single plot.
\item
  For all $\gamma$ in the given range, the Jacobian matrix at the
  solution remains negative definite.  Plot
  $\lambda_{\max}(J(v^*))$ (the eigenvalue closest to zero) as a
  function of $\gamma$.  What do you notice?
\item
  Try running your code again, this time going up to a maximum value
  of 4 rather than 3.5.  What happens?
\end{itemize}
Note: You may start from the following code
\begin{lstlisting}
  n = 100;
  h = 1/(n+1);
  T = (n+1)^2 * (diag(ones(n-1,1),-1) + diag(one(n-1,1),1) - 2*eye(n));
  v = zeros(n,1);
\end{lstlisting}
If $n$ was very large, we might want to use a sparse matrix\footnote{%
I'd probably switch to a more accurate discretization method, first,
but that's a topic for CS 4210.
}, but it's probably not worth it in this case.

\end{document}
