\documentclass[12pt, leqno]{article}
\usepackage{amsfonts}
\usepackage{amsmath}
\usepackage{fancyhdr}
\usepackage{hyperref}
\usepackage{tikz}
\usepackage{pgfplots}
\usepackage{listings}

\newcommand{\bbR}{\mathbb{R}}
\newcommand{\bbC}{\mathbb{C}}

\newcommand{\hdr}[2]{
  \pagestyle{fancy}
  \lhead{Bindel, Spring 2015}
  \rhead{Numerical Analysis (CS 4220)}
  \fancyfoot{}
  \begin{center}
    {\large{\bf #1}} \\
    Due: #2
  \end{center}
  \lstset{language=matlab,columns=flexible}
}

\newcommand{\phdr}[1]{
  \pagestyle{fancy}
  \lhead{Bindel, Spring 2015}
  \rhead{Numerical Analysis (CS 4220)}
  \fancyfoot{}
  \begin{center}
    {\large{\bf #1}}
  \end{center}
  \lstset{language=matlab,columns=flexible}
}


\begin{document} \hdr{PS 3}{Weds, Feb 11}

\paragraph*{1: By the book}
Book section 2.5, p10, 11; section 4.6, p16.

\paragraph*{2: Definitions}
Let $\hat{x} = 32$ be regarded as an approximation to the positive
solution for $f(x_*) = x_*^2 - 1000 = 0$.  What are the absolute
error, the relative error, and the residual error?

\paragraph*{3: Pi, see!}
The following routine estimates $\pi$ by recursively computing the
semiperimeter of a sequence of $2^{k+1}$-gons embedded in the unit circle:
\lstset{language=matlab,frame=lines,columns=flexible}
\lstinputlisting{ps3pibad.m}
Plot the absolute error $|s_k-\pi|$ against $k$ on a semilog plot.
Explain why the algorithm behaves as it does, and describe a
reformulation of the algorithm that does not suffer from this problem.

\end{document}
