\documentclass[12pt, leqno]{article}
\usepackage{amsthm}
\usepackage{amsfonts}
\usepackage{amsmath}
\usepackage{fancyhdr}
\usepackage{hyperref}
\usepackage{tikz}
\usepackage{pgfplots}
\usepackage{listings}

\newcommand{\bbR}{\mathbb{R}}
\newcommand{\bbC}{\mathbb{C}}

\newcommand{\hdr}[2]{
  \pagestyle{fancy}
  \lhead{Bindel, Spring 2015}
  \rhead{Numerical Analysis (CS 4220)}
  \fancyfoot{}
  \begin{center}
    {\large{\bf #1}} \\
    Due: #2
  \end{center}
  \lstset{language=matlab,columns=flexible}
}

\newcommand{\phdr}[1]{
  \pagestyle{fancy}
  \lhead{Bindel, Spring 2015}
  \rhead{Numerical Analysis (CS 4220)}
  \fancyfoot{}
  \begin{center}
    {\large{\bf #1}}
  \end{center}
  \lstset{language=matlab,columns=flexible}
}


\begin{document} \phdr{Practice Midterm}

This is around the right scale and form for a midterm for 4220, but it may
be somewhat harder or easier than the actual midterm.  The actual exam
will be open book and notes, but limited to 50 minutes.  In evaluating
yourself, you may want to try the exam under those conditions.

\paragraph*{1: True/False}
\begin{enumerate}
\item
  Suppose $Ax = b$ and $(A+E) \hat{x} = b$.
  Then $\|\hat{x}-x\| \leq \kappa(A) \|E\|$.
\item
  If $a$ and $b$ are normalized floating point numbers and $a+b$ is in
  the range of normalized floating point numbers, then
  $\fl(a+b) = (a+b)(1+\delta)$ where $|\delta| \leq \macheps$.
\item
  Newton's iteration is quadratically convergent for $f(x) = x^2 = 0$
  for starting points sufficiently near zero.
\item
  If $A$ is singular, Gaussian elimination cannot compute $PA = LU$.
\item
  In Gaussian elimination with partial pivoting, all elements of $L$
  below the main diagonal have magnitude at most one.
\end{enumerate}

{\bf Answer:}
\begin{enumerate}
\item {\bf False}.  The condition number relates relative errors, not
  absolute errors.
\item {\bf True}.
\item {\bf False}.  In this case, Newton's iteration is $x_{k+1} =
  x_k/2$, which converges linearly.  We saw something like this in a
  homework.
\item {\bf False}.  Singular matrices can have $LU$ factorizations;
  consider
  \[
  L = \begin{bmatrix} 1 & 0 \\ 0 & 1 \end{bmatrix}, \quad
  U = \begin{bmatrix} 1 & 1 \\ 0 & 0 \end{bmatrix}
  \]
\item {\bf True}.  This is the point of partial pivoting.
\end{enumerate}

\paragraph*{2: Fixed point fandango}
Consider the iteration
\[
  x_{k+1} = 10-\exp(x_k).
\]
The iteration has a fixed point $x_* \approx 2.0706$.  For $x_0$
close enough to $x_*$, does the iteration converge?  Explain
by writing an error recurrence.

{\bf Answer:}
The iteration equation is
\[
  e_{k+1} = -(\exp(x_*+e_k)-\exp(x_*)) = -\exp(x_*) e_k + O(e_k^2)
\]
Because $|\exp(x_*)| > 1$, the iteration diverges in general.

\paragraph*{3: Norm!}
The {\em Frobenius norm} of a matrix $A$ is
\[
  \|A\|_F = \sqrt{ \sum_{i,j} a_{ij}^2 }.
\]
Show
\begin{enumerate}
\item
  The Frobenius norm is not an operator norm (hint: consider
  $\|I\|_F$).
\item
  The Frobenius norm is consistent with the two norm, i.e.
  \[
    \|Av\|_2 \leq \|A\|_F \|v\|_2.
  \]
  {\em Hint:} The Cauchy-Schwarz inequality states $|x \cdot y| \leq
  \|x\|_2 \, \|y\|_2$.
\end{enumerate}

{\bf Answer:}
\begin{enumerate}
\item
  For an operator norm,
  \[
    \|I\| = \max_{x \neq 0} \|x\|/\|x\| = 1.
  \]
  For the Frobenius norm, $\|I\|_F = \sqrt{n}$.
\item
  By Cauchy-Schwarz,
  \[
    (Av)_i^2 \leq \|A(i,:)\|_2^2 \|v\|_2^2
  \]
  Summing over $i$, we have
  \[
    \|Av\|_2^2 \leq \|A\|_F^2 \|v\|_2^2
  \]
\end{enumerate}

\paragraph*{4: Pseudoinverse}
Suppose $A \in \bbR^{n \times m}$ has full column rank, $n > m$.
\begin{enumerate}
\item
  Write the pseudoinverse in terms of $A$, the economy QR
  factorization of $A$, and the economy SVD of $A$.
\item
  Show that if $n = m$, then the pseudoinverse is the same as the
  inverse.
\item
  Give a {\em brief} geometric characterization of the null space of
  $A^\dagger$.
\end{enumerate}

{\bf Answer:}
\begin{enumerate}
\item $A^\dagger = (A^TA)^{-1} A^T = R^{-1} Q^T = V \Sigma^{-1} U^T$
\item $A^\dagger = (A^T A)^{-1} A^T = A^{-1} A^{-T} A^T = A^{-1}$ when
  $A$ (and hence $A^T$) are square and invertible.
\item The null space of $A^{\dagger}$ is those vectors orthogonal
  to the range space of $A$.
\end{enumerate}

\paragraph*{5: Elimination and low rank}
Consider the matrix
\[
  A = I + uv^T
\]
where $u$ and $v$ are elementwise positive, $\|u\|_1 < 1$ and $\|v\|_1 < 1$.
\begin{enumerate}
\item
  $A$ must be diagonally dominant.  Briefly state why.
\item
  Show that after one step of Gaussian elimination,
  the Schur complement has the form
  \[
    S = I + \alpha u_2 v_2^T.
  \]
  Write a simple expression for the coefficient $\alpha$.
\end{enumerate}

{\bf Answer:}
\begin{enumerate}
\item
  Each column is $e_j + u v_j$.  The $j$th element is $1 + u_j > 1$;
  the sum of the remaining element magnitudes is bounded by 
  $\sum_i |u_i v_j| = \|u\|_1 |v_j| < 1$.
\item
  One step of Gaussian elimination gives
  \[
    S = I + \frac{u_1 v_1}{1 + u_1 v_1} u_2 v_2^T
  \]
\end{enumerate}

\end{document}
